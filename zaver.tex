\section*{Závěr}
Měřili jsme  závislost povrchového napětí na koncentraci lihu v roztoku.
Změřili jsme povrchové napětí destilované vody (\SI{82(3)}{\num{e-3}\,\si{\newton \per \metre}}) a čistého lihu (\SI{30(2)}{\num{e-3}\,\si{\newton \per \metre}}) při teplotě \SI{23.5(3)}{\degreeCelsius}.
Tyto hodnoty se sice příliš neshodují s tabelovanými, ale obě se od nich liší přibližně o stejnou hodnotu, takže můžeme předpokládat, že naše měření bylo zatíženo stálou systematickou chybou.
To znamená, že se nám sice nepodařilo změřit spolehlivě přímo povrchové napětí, nicméně klesající trend (viz graf \ref{grp::graf}) při zvyšování koncentrace lihu považujeme za velmi správný.
Při nízkých koncentracích má přidání i malého objemu lihu za následek rapidní pokles povrchového napětí, při vyšších koncentracích již efekt není příliš patrný.