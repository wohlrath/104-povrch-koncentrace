\section*{Diskuze}
Zdroj \cite{napetivoda} uvádí při \SI{20}{\degreeCelsius} povrchové napětí destilované vody (\num{72.75(36)})\,\SI{e-3}{\newton\per\metre} a při \SI{30}{\degreeCelsius} napětí\linebreak (\num{71.99(36)})\,\SI{e-3}{\newton\per\metre}.
Naše hodnota 
\SI{74(3)}{\num{e-3}\,\newton \per \metre}
při \SI{23.5}{\degreeCelsius} se v rámci směrodatné odchylky s těmito hodnotami shoduje.

Zdroj \cite{napetiethanol} uvádí pro teploty okolo \SI{20}{\degreeCelsius} povrchové napětí ethanolu $(\num{22.10} - \num{0.0832}(t- \SI{20}{\degreeCelsius}))$\SI{e-3}{\newton\per\metre}, pro teplotu \SI{23.5}{\degreeCelsius} tedy přibližně \SI{21.8}{\num{e-3}\,\newton \per \metre}.
Naše hodnota \SI{25(2)}{\num{e-3}\,\newton \per \metre} se s ní shoduje v rámci dvou směrodatných odchylek.

Při mísení lihu s vodou dochází k exotermické reakci a směs se zahřívá.
Po smíchání jsme před měřením vždy chvíli počkali, ale směs mohla mít stále nepatrně vyšší teplotu.
Povrchové napětí je na teplotě závislé, chybu způsobenou změnou teploty ale považujeme za menší než chybu způsobenou nepřesností měření, náhodnými jevy a systematickou chybu metody a zanedbáváme ji. 

Další efekt, který mohl způsobit chybu, je kontrakce objemu při mísení lihu s vodou.
Po dvojnásobném zředení v poměru $1:1$ totiž nevznikne roztok o stejné koncentraci jako po jednom zředění v poměru $1:3$.
Tuto chybu však považujeme za zanedbatelnou s ohledem na další nepřesnosti při mísení.

Použitý líh samozřejmě nebyl stoprocentní.
Uvedené koncentrace pouze vyjadřují poměr, ve kterém jsme ho mísili s destilovanou vodou.