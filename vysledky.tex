\section*{Výsledky měření}
Teplota v místnosti byla \SI{23.5(5)}{\degreeCelsius}.
Teplota destilované vody i lihu byla před smícháním stejná.

Drátek v rámečku jsme měřili posuvným měřítkem a měl délku $l = \SI{1.94(5)}{\cm}$.

Na váhy jsme pověsili přívažek o hmotnosti \SI{200}{\milli\g}.
Jelikož síly od sebe odečítáme, nemá přívažek na výsledek žádný vliv.

Torzními vahami jsme přímo měřili rozdíl hmotností na obou ramenech.
Ke změření síly je tedy nutný přepočet na tíhovou sílu $F = mg$, tedy
\begin{equation}
P_1 = m_1 \cdot g \qquad P_2= m_2 \cdot g \,,
\end{equation}
kde $m_1$, $m_2$ jsou váhami změřené rozdíly hmotností na ramenech, když je rámeček těsně pod vodou resp. právě se odtrhl od hladiny, a $g$ je tíhové zrychlení.
Tíhové zrychlení v Praze je \SI{9,814}{\m\per\s\squared} \cite{gravitace}, chybu zanedbáváme.

Vzorec \eqref{eq::p0sigma} má po úpravě a dosazení tvar
\begin{equation}
\sigma = \frac{(m_2-m_1)\cdot g}{2 \cdot l}
\end{equation}

Nejdříve jsme změřili povrchové napětí čisté destilované vody a čistého lihu.
Poté jsme s pomocí pyknometru připravili roztok s \SI{50}{\percent} koncentrací lihu a tento roztok jsme dále ředili vodou vždy v poměru $1:1$.

V tabulce \ref{tab::namereny} jsou pro měřené koncentrace uvedeny hodnoty $m_1$, $m_2$, vypočtené síly $P_0$ a povrchové napětí~$\sigma$.
U $m_1$ a $m_2$ uvádíme přímo číselný údaj na vahách, pro výpočet celkové hodnoty $P_1$ nebo $P_2$ by bylo nutné přičíst hmotnost přívažku.
Směrodatnou odchylku $\mu_m$\footnote{Symbol $\mu$ pro směrodatnou odchylku používáme, aby nedošlo k záměně za povrchové napětí $\sigma$.} určení rozdílu hmotností $(m_2-m_1)$ odhadujeme na \SI{5}{\milli \g}.
Z toho se odvíjí směrodatná odchylka síly $\mu_{P_0} = \SI{0.05}{\milli \newton}$.
Směrodatnou odchylku povrchového napětí počítáme jako
\begin{equation}
\mu_\sigma = \sigma \sqrt{ \left(  \frac{\mu_{P_0}}{P_0} \right)^2 +
\left(  \frac{\mu_{l}}{l} \right)^2   }
\end{equation}

Závislost povrchového napětí na koncentraci lihu v roztoku je zanesena do grafu \ref{grp::graf}.

\begin{tabulka}[htbp]
\centering
\begin{tabular}{ccccc}
koncentrace lihu & $m_1 (\si{\milli\g})$ & $m_2 (\si{\milli\g})$ & $P_0 (\si{\milli \newton})$ & $\sigma (\num{e-3}\,\si{\newton\per\metre})$ \\ \hline 
\SI{0}{\percent} & \num{125} & \num{448} & \num{3.17} & \num{82(3)} \\
\SI{0.4}{\percent} & \num{118} & \num{430} & \num{3.06} & \num{79(3)} \\
\SI{0.8}{\percent} & \num{117} & \num{419} & \num{2.96} & \num{76(3)} \\
\SI{1.6}{\percent} & \num{113} & \num{404} & \num{2.86} & \num{74(3)} \\
\SI{3.1}{\percent} & \num{116} & \num{401} & \num{2.80} & \num{72(3)} \\
\SI{6.3}{\percent} & \num{111} & \num{370} & \num{2.54} & \num{66(3)} \\
\SI{12.5}{\percent} & \num{109} & \num{327} & \num{2.14} & \num{55(2)} \\
\SI{25}{\percent} & \num{106} & \num{277} & \num{1.68} & \num{43(2)} \\
\SI{50}{\percent} & \num{105} & \num{238} & \num{1.31} & \num{34(2)} \\
\SI{100}{\percent} & \num{106} & \num{225} & \num{1.17} & \num{30(2)} \\
\end{tabular}
\caption{Naměřené hodnoty povrchového napětí pro různé koncentrace lihu v roztoku}
\label{tab::namereny}
\end{tabulka}

\begin{graph}[htbp] 
\centering
\input{graf.tex}
\caption{Závislost povrchového napětí na koncentraci lihu v roztoku}
\label{grp::graf}
\end{graph}